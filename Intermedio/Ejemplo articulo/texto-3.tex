\documentclass{article}
%diagonal par diagonal newline diferencias
\usepackage{amsmath}
\usepackage{amssymb}
\usepackage{graphicx}
\usepackage[spanish, es-tabla]{babel}
\usepackage{xcolor}
\usepackage{float}
\usepackage{booktabs}

\begin{document}
\title{Teor\'ia y uso de los logaritmos.}
\author{Luis Ignacio Garcia Reyes}

\maketitle

\textbf{Necesidad de los logaritmos en la trigonometr\'ia.}\par
Muchos de los problemas que se presentan en la Trigonometr\'ia implican largos c\'alculos.\newline Como el trabajo relacionado con ellos puede verse aminorado grandemente con el uso de los logaritmos, resulta ventajoso usarlos en una gran parte de los c\'alculos trigonom\'etricos.
Esto es  especialmente cierto en los c\'alculos relacionados con la soluci\'on de los tri\'angulos.\par A continuaci\'on vamos a dar los principios fundamentales de los logaritmos y a explicar el uso de las tablas de logaritmos.
\begin{quote}
Debemos de resaltar la diferencia entre \textbackslash par y \textbackslash newline de \LaTeX.
\end{quote}

\end{document}
\documentclass{article}

\usepackage{amsmath}
\usepackage{amssymb}
\usepackage{graphicx}
\usepackage{booktabs}
\begin{document}
\title{Tablas}
\author{Luis Ignacio Garcìa Reyes}
\maketitle
% tabular environment creates tables
% & means "line up here"
% \\ means "line break here"
Aqui vamos a hacer referencia a como formar una tabla y algunos estilos diferentes para las mismas.\newline\newline
Introducimos la funcion $f(x)=\sqrt{x^2+1}$ y damos los siguientes valores:\newline\newline
\begin{center}
\begin{tabular}{r|cccc}

$x$ &-1&0&1&2\\
\hline
$f(x)$  &$\sqrt{2}$ &1& $\sqrt{2}$ & $\sqrt{5}$

\end{tabular}

\end{center}
Se muestra evidente la diferencia entre una y otra.
\begin{center}
\begin{tabular}{|c|c|c|c|c|} \hline
x&-1&0&1&2 \\ \cline{1-5}
$f(x)$&$\sqrt{2}$&1&$\sqrt{2}$&$\sqrt{5}$\\ \hline
\end{tabular}
\end{center}
	Aqui otro ejemplo de tabla.
\begin{center}
\begin{tabular}{llllr} \toprule
$x$ &-1&0&1&2 \\ \midrule
$f(x)$  &$\sqrt{2}$ &1& $\sqrt{2}$ & $\sqrt{5}$\\ \bottomrule
\end{tabular}
\end{center}
Tambien podemos hacer una serie de ediciones sobre las tablas, como rotarlas, unir celdas unir columnas o renglones etc.
\end{document}
\documentclass{book}
\usepackage{amsmath}
\usepackage{amssymb}
\usepackage{xcolor}

\begin{document}
\title{caja de color}
\author{Luis Ignacio Garcia Reyes}
\maketitle

\begin{center}{
\fboxsep 20pt
\fcolorbox {orange}{white}{
\begin{minipage}[t]{10cm}

Programa que genera y grafica un impulso unitario\newline
tmin=input('valor minimo del tiempo: ');\newline
tmax=input('valor maximo del tiempo: ');\newline
A=input('Intensidad del impulso: ');\newline
Np=input('Numero total de puntos: ');\newline
Nme=input('Numero de multiplos enterosde pi para graficar la transformada: ');\newline
t=tmin:(tmax-tmin)/Np:tmax;\newline
origen=find(t==0);\newline
f=zeros(1,length(t));\newline
f(origen)=1*A;\newline
subplot(2,1,1);\newline
stem(t,f,'.');\newline
xlabel('t');\newline
ylabel('f(t)');\newline
title('Impulso unitario.');\newline
axis([min(t) max(t) -0.1*max(f) 1.1*max(f)]);\newline
subplot(2,1,2);\newline
W=-Nme*pi:2*Nme*pi/Np:Nme*pi;\newline
F=A*ones(1,length(W));\newline
plot(W,F);\newline
xlabel('W');\newline
ylabel('f(W)');\newline
title('Transformada de Fourier del impulso.');\newline
clear all
clc

\end{minipage}
}}
\end{center}





\end{document}
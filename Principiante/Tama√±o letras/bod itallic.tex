\documentclass{article}

\usepackage{amsmath} %para insertar ecuaciones
\usepackage{amssymb}
\usepackage[spanish]{babel}
\usepackage{mathrsfs} %solo lee mayusculas
\usepackage[mathscr]{eucal}%combinacion del paquete anterior se sugiere borrar para que se vea el cambio en la ultima linea
\begin{document}


\title{Estilos de letras y tama\~no.}


\author{Luis Ignacio  Garc\'ia}
\maketitle
% Hello
\sc{Tipos de fuente}\newline\newline %small caps
\begin{centering}

\textbf{world},\newline %negrita
\textit{here}\newline %italica
\em{ is my}\newline %enfatico
\sf{first document}\newline  %sans serif
\texttt{first}\newline %typewritter
\underline{ document!!!}\newline %subrayado
\end{centering}
\newline
%this is a coment
this is not a coment\newpage

\sc{Tama\~nos de Letras} en \LaTeX \newline\newline
\begin{centering}

\tiny{Diversi\'on}\newline
\scriptsize{Diversi\'on}\newline
\footnotesize{Diversi\'on}\newline
\small{Diversi\'on}\newline
\normalsize{Diversi\'on}\newline
\large{Diversi\'on}\newline
\Large{Diversi\'on}\newline
\huge{Diversi\'on}\newline
\Huge{Diversi\'on}\newline
\end{centering}
\newpage
Fuentes en modo matem\'atico en \LaTeX:

$$\mathbb{texto}$$
$$\mathbf{Texto}$$
$$\mathcal{Texto}$$
$$\mathfrak{Texto}$$
$$\mathit{texto}$$
$$\mathnormal{Texto}$$
$$\mathrm{Texto}$$
$$\mathtt{Texto}$$
$$\mathscr{TEXTO}$$




\end{document}